% *************** Style table de matière et listes de figures ***************
\setsecnumdepth{subsubsection}
\maxsecnumdepth{subsubsection}
\settocdepth{subsubsection}
\maxtocdepth{subsubsection}

\renewcommand{\chapternumberline}[1]{
% Impression du texte chapitre ou annexe
\hspace{-0.4cm}\textbf{
	$\mathit{\myPrintChapterLabel{#1}}$
	$\mathpzc{{#1}}~\RHD$}
}

\renewcommand{\partnumberline}[1]{
% Impression du texte partie
\hspace{-0.4cm}\textbf{
	\myPartLabel $\,$ #1 $\,$--$\,$}
}

\newcommand{\lof}{false}
\renewcommand{\numberline}[1]{
\ifthenelse{\equal{\lof}{false}}{\hspace{-0.6cm}$\mathrm{{#1}}$ -} {$\mathrm{{#1}}$ -}
}

\let\oldcontentsline=\contentsline
\renewcommand{\contentsline}[4]{
	\vspace{0.8mm}
	\oldcontentsline{#1}{#2}{#3}{#4}
	\vspace{0.8mm}
}
